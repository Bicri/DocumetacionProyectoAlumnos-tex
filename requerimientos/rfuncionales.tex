\section{Requerimientos funcionales}


A continuación, se presenta la lista correspondiente a los requerimientos funcionales del sistema. Asignándole a cada requerimiento un identificador, un nombre y una breve descripción.\\


\textbf{RF1 \hspace{2cm}Tipo de sistema}\\
El sistema es una aplicación de escritorio que se ejecuta sin alguna dependencia externa de otro equipo de cómputo o conexión a internet.\\

\textbf{RF2 \hspace{2cm}Sistema Operativo}\\
La aplicación debe ejecutarse sobre el sistema operativo Windows 10.\\

\textbf{RF3 \hspace{2cm}Selección de operación primaria}\\
El sistema debe mostrar una opción de visualizar alumnos y otra de insertar alumnos al inicio de la aplicación.\\

\textbf{RF4 \hspace{2cm}Registrar Alumno}\\
El sistema debe permitir al profesor registrar nuevos alumnos a su lista. Los datos que se deben proporcionar son:
\begin{itemize}
	\item Nombre o Nombres
	\item Primer apellido
	\item Segundo apellido \textit{(opcional)}
	\item Matricula
\end{itemize}
Siendo el Segundo apellido el campo opcional. De la misma forma de aceptar el ingreso de cualquier caracter para nombres y apellidos\\

\textbf{RF5 \hspace{2cm}Editar Alumno}\\
El sistema debe permitir al profesor modificar cualquier dato de algún alumno que se encuentre inscrito\\

\textbf{RF6 \hspace{2cm}Eliminar Alumno}\\
El sistema debe permitir al profesor eliminar cualquier alumno que se encuentre registrado\\

\textbf{RF7 \hspace{2cm}Validar Matrículas}\\
El sistema no debe permitir la existencia de dos matrículas iguales\\

\textbf{RF4 \hspace{2cm}Sin existencia de Alumnos}\\
El sistema debe notificar cuando no haya registros y se quiera eliminar un alumno\\








